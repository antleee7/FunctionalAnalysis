\documentclass{beamer}
\usetheme{CambridgeUS}  % Select your favorite theme
\title{Convexity}
\author{Chapter 3}
\institute{Functional Analysis}
\date{\today}
\begin{document}
\begin{frame}
\titlepage
\end{frame}
\section{The Hahn-Banach Theorems}
\begin{frame}
\frametitle{Dual space}
\begin{definition}
    The \textbf{dual space} of a TVS $X$ is the vector space $X^*$ whose elements are the continuous linear functionals on $X$.
\end{definition}
Note that addition and scalar multiplication are defined in $X^*$ by $(\Lambda_1 + \Lambda_2)x = \Lambda_1 x + \Lambda_2 x$ and $(\alpha\Lambda)x = \alpha \cdot \Lambda x$. Thus it is clear that $X^*$ is indeed a vector space.
\end{frame}
\begin{frame}
\frametitle{Some terminology}
An additive functional $\Lambda$ on a complex vector space $X$ is called real-linear (resp. complex-linear) if $\Lambda(\alpha x) = \alpha \Lambda x$ for all $x \in X$ and for every real (resp. complex) scalar $\alpha$. \newline\newline Note that if $u$ is the real part of a complex-linear functional $f$ on $X$, then $u$ is real-linear and $f(x) = u(x) - iu(x)$ because $z = \text{Re}z - i\text{Re}(iz)$ for all $z\in \mathbb{C}$. Conversely, if $u:X \to \mathbb{R}$ is real-linear on a complex vector space $X$ and if $f$ is defined the same, then $f$ is also complex-linear. \newline\newline Thus we may conclude that a complex-linear functional on $X$ is an element of $X^*$ if and only if its real part is continuous, and that every continuous real-linear $u:X \to \mathbb{R}$ is the real part of a unique $f\in X^*$.
\end{frame}
\end{document}
