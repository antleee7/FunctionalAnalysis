\documentclass{beamer}
\usetheme{CambridgeUS}  % Select your favorite theme
\title{Convexity}
\author{Chapter 3}
\institute{Functional Analysis}
\date{\today}
\begin{document}
\begin{frame}
\titlepage
\end{frame}
\section{The Hahn-Banach Theorems}
\begin{frame}
\frametitle{Dual space}
\begin{definition}
    The \textbf{dual space} of a TVS $X$ is the vector space $X^*$ whose elements are the continuous linear functionals on $X$.
\end{definition}
Note that addition and scalar multiplication are defined in $X^*$ by $(\Lambda_1 + \Lambda_2)x = \Lambda_1 x + \Lambda_2 x$ and $(\alpha\Lambda)x = \alpha \cdot \Lambda x$. Thus it is clear that $X^*$ is indeed a vector space.
\end{frame}
\begin{frame}
\frametitle{Some terminology}
An additive functional $\Lambda$ on a complex vector space $X$ is called real-linear (resp. complex-linear) if $\Lambda(\alpha x) = \alpha \Lambda x$ for all $x \in X$ and for every real (resp. complex) scalar $\alpha$. \newline\newline Note that if $u$ is the real part of a complex-linear functional $f$ on $X$, then $u$ is real-linear and $f(x) = u(x) - iu(x)$ because $z = \text{Re}z - i\text{Re}(iz)$ for all $z\in \mathbb{C}$. Conversely, if $u:X \to \mathbb{R}$ is real-linear on a complex vector space $X$ and if $f$ is defined the same, then $f$ is also complex-linear. \newline\newline Thus we may conclude that a complex-linear functional on $X$ is an element of $X^*$ if and only if its real part is continuous, and that every continuous real-linear $u:X \to \mathbb{R}$ is the real part of a unique $f\in X^*$.
\end{frame}
\begin{frame}
\frametitle{Dominated extension theorems}
\begin{theorem}
    Suppose $M$ is a subspace of a real vector space $X$, $p:X\to \mathbb{R}$ satisfies $p(x+y)\leq p(x) + p(y)$ and $p(tx) = p(x)$ for $x,y\in X$ and $t\geq 0$, and $f:M\to \mathbb{R}$ is linear and $f(x) \leq p(x)$ on $M$. Then, there exists a linear $\Lambda : X \to \mathbb{R}$ such that $\Lambda x = f(x)$ for $x\in M$, and $-p(-x) \leq \Lambda x \leq p(x)$ for all $x \in X$.
\end{theorem}
\end{frame}
\begin{frame}
\frametitle{Dominated extension theorems}
\begin{proof}\renewcommand{\qedsymbol}{}
    Suppose $M \neq X$. Choose $x_1 \in X \setminus M$, and define $M_1 = \{x+tx_1\mid x\in M,\quad t\in \mathbb{R}\}$. Check that $M_1$ is a vector space. Since $f(x) + f(y) = f(x+y) \leq p(x+y) \leq p(x-x_1) + p(x_1+y)$, we have $f(x) - p(x-x_1) \leq p(y+x_1) - f(y)$ for $x,y\in M$. Let $\alpha$ be the least upper bound of $f(x) - p(x-x_1)$ as $x$ goes through $M$. Then, $f(x) -\alpha \leq p(x-x_1)$ for all $x \in M$, and $f(y) + \alpha \leq p(y+x_1)$ for all $y \in M$. \newline\newline Now we define $f_1$ on $M_1$ by $f_1(x+tx_1) = f(x)+t\alpha$ for $x \in M$ and $t\in \mathbb{R}$. Then, $f_1 = f|_M$ and $f_1$ is linear on $M_1$. Also $f_1 \leq p$ on $M_1$.\footnote{??}
\end{proof}
\end{frame}
\begin{frame}
\frametitle{Dominated extension theorems}
\begin{proof}\renewcommand{\qedsymbol}{}
    Now we conclude the proof by transfinite induction. Let $\mathcal{P}$ be the collection of all pairs $(M^\prime,f^\prime)$ where $M^\prime$ is a subspace of $X$ containing $M$, and $f^\prime$ is a linear functional on $M^\prime$ extending $f$, and satisfying $f^\prime \leq p$ on $M^\prime$. Impose a partial ordering by $(M^\prime,f^\prime) \leq (M^{\prime\prime},f^{\prime\prime})$ if $M^\prime \subset M^{\prime\prime}$ and $f^\prime = f^{\prime\prime}$ on $M^\prime$. By Hausdorff's maximality theorem, there exists a maximal totally ordered subcollection $\Omega \subset \mathcal{P}$.\newline\newline Let $\Phi$ be the collection of all $M^\prime$ such that $(M^\prime,f^\prime) \in \Omega$. This makes $\Phi$ a totally ordered set, and the union $\tilde{M}$ of all members of $\Phi$ is a subspace of $X$.\footnote{Why?} Note that if $x \in \tilde{M}$ then $x\in M^\prime$ for some $M^\prime \in \Phi$, so we define $\Lambda x = f^\prime(x)$ where $f^\prime$ is the pair function of $M^\prime$.
\end{proof}
\end{frame}
\begin{frame}
\frametitle{Dominated extension theorems}
\begin{proof}
    Note that $\Lambda$ is well-defined on $\tilde{M}$.\footnote{Not so clear..} Also $\Lambda$ is linear, and $\Lambda \leq p$. Now if $\tilde{M}$ were a proper subspace of $X$, then we could extend $\tilde{M}$, which contradicts the maximality of $\Omega$. Thus $\tilde{M} = X$ and we have $-p(-x) \leq -\Lambda(-x) = \Lambda x$ for all $x\in X$. 
\end{proof}
\end{frame}
\begin{frame}
\frametitle{Dominated extension theorems}
Here is another dominated extension theorem.
\begin{theorem}
    Suppose $M$ is a subspace of a vector space $X$. Let $p$ be a seminorm on $X$, and $f$ be a linear functional on $M$ such that $|f(x)| \leq p(x)$ for all $x\in M$. Then $f$ extends to a linear functional $\Lambda$ on $X$ such that $|\Lambda x| \leq p(x)$ for all $x\in X$.
\end{theorem}
\begin{proof}\renewcommand{\qedsymbol}{}
    If the scalar field is $\mathbb{R}$, then this follows by the previous theorem since $p(-x) = p(x)$. Now assume the scalar field to be $\mathbb{C}$. Put $u = \text{Re}f$. By the previous theorem, there is a real-linear $U$ on $X$ such that $U = u$ on $M$ and $U \leq p$ on $X$.
\end{proof}
\end{frame}
\begin{frame}
\frametitle{Dominated extension theorems}
\begin{proof}
    Let $\Lambda$ be the complex-linear functional on $X$ whose real part is $U$. Now it follows that $\Lambda = f$ on $M$. Finally, to each $x \in X$ there is an $\alpha \in \mathbb{C}$, $|\alpha| = 1$ such that $\alpha\Lambda x = |\Lambda x|$. Hence \[|\Lambda x| = \Lambda(\alpha x) = U(\alpha x) \leq p(\alpha x) = p(x)\]
\end{proof}
\begin{corollary}
    If $X$ is a normed space and $x_0 \in X$, then there exists $\Lambda \in X^*$ such that $\Lambda x_0 = ||x_0||$ and $|\Lambda x| \leq ||x||$ for all $x\in X$.
\end{corollary}
\end{frame}
\begin{frame}
\frametitle{Dominated extension theorems}
\begin{proof}
    For $x_0 = 0$, take $\Lambda = 0$. If $x_0 \neq 0$, apply the previous theorem with $p(x) = ||x||$ and $M$ the one-dimensional space generated by $x_0$, and $f(\alpha x_0) = \alpha||x_0||$.
\end{proof}
\end{frame}
\begin{frame}
\frametitle{The separation theorem}
\begin{theorem}
    Suppose $A,B$ are disjoint nonempty convex sets in a TVS $X$.\newline\newline \textbf{(a)} If $A$ is open, then there exist $\Lambda \in X^*$ and $\gamma \in \mathbb{R}$ such that $\text{Re}\Lambda x < \gamma \leq \text{Re}\Lambda y$ for every $x \in A$ and $y \in B$.\newline\newline \textbf{(b)} If $A$ is compact, $B$ is closed and $X$ is locally convex, then there exist $\Lambda \in X^*$, $\gamma_1 \in \mathbb{R}$ and $\gamma_2 \in \mathbb{R}$ such that $\text{Re}\Lambda x < \gamma_1 < \gamma_2 < \text{Re}\Lambda y$ for all $x \in A$, $y \in B$.
\end{theorem}
Note that if the scalar field is $\mathbb{R}$, then Re$\Lambda = \Lambda$.
\end{frame}
\begin{frame}
\frametitle{The separation theorem}
\begin{proof}
    We let the scalar field be $\mathbb{R}$ since for $\mathbb{C}$ there is a unique complex-linear functional on $X$ whose real part is the desired real-linear functional.\newline\newline \textbf{(a)} Fix $a_0 \in A$ and $b_0 \in B$. Put $x_0 = b_0-a_0$ and put $C = A-B + x_0$. T
\end{proof}
%TODO: Write proof, finish section
\end{frame}
\begin{frame}
\frametitle{}
\end{frame}

\section{Weak Topologies}
\begin{frame}
\frametitle{Weak topologies}
Let $\mathcal{T}_1$ and $\mathcal{T}_2$ be two topologies on a set $X$, and assume $\mathcal{T}_1 \subset \mathcal{T}_2$. Then we say $\mathcal{T}_1$ is weaker than $\mathcal{T}_2$, or stronger vice versa.\newline\newline Notice that the topology of a compact Hausdorff space has rigidity; It cannot be weakened without losing the Hausdorff axiom, and cannot be strengthened without losing compactness.\begin{fact}
    If $\mathcal{T}_1 \subset \mathcal{T}_2$ on $X$, and if $\mathcal{T}_1$ is Hausdorff and $\mathcal{T}_2$ is compact, then $\mathcal{T}_1 = \mathcal{T}_2$.
\end{fact}
To show this, consider a $\mathcal{T}_2$-closed $F \subset X$. Thus $F$ is compact in $\mathcal{T}_2$, from which it follows that it is compact in $\mathcal{T}_1$, hence is $\mathcal{T}_1$-closed. Therefore $\mathcal{T}_1 \supset \mathcal{T}_2$, so the two are equal.
\end{frame}
\begin{frame}
\frametitle{Weak topologies}
Here is another fact.\begin{fact}
    Consider the quotient topology $\mathcal{T}_N$ of $X/N$, where $E \in \mathcal{T}_N$ if $\pi^{-1}(E) \in \mathcal{T}$. By definition, $\mathcal{T}_N$ is the strongest topology on $X/N$ that makes $\pi$ continuous, and is the weakest one that makes $\pi$ an open mapping.
\end{fact}
\end{frame}
\begin{frame}
\frametitle{Weak topologies}
\begin{Definition}
    Suppose $X$ is a set and $\mathcal{F}$ is a nonempty family of mappings $f: X \to Y_f$ where each $Y_f$ is a topological space. Let $\mathcal{T}$ be the topology generated by subbases $f^{-1}(V)$ for $f \in \mathcal{F}$ and $V$ open in $Y_f$. Then $\mathcal{T}$ is the weakest topology on $X$ that makes each $f$ continuous. This $\mathcal{T}$ is called the \textbf{weak topology} on $X$ induced by $\mathcal{F}$, or the $\mathcal{F}$-topology of $X$.
\end{Definition}
\begin{example}
    The product topology $\mathcal{T}$ of $X = \prod X_\alpha$ is the $\{\pi_\alpha\}$-topology of $X$, the weakest one that makes each projection map continuous.
\end{example}
\end{frame}
\begin{frame}
\frametitle{Weak topologies}
\begin{fact}
    Suppose $\mathcal{F}$ is a family of mappings $f:X \to Y_f$, where $X$ is a set and each $Y_f$ is a Hausdorff space. If $\mathcal{F}$ separates points on $X$, then the $\mathcal{F}$-topology of $X$ is Hausdorff.
\end{fact}
\begin{proof}
    Suppose $p \neq q$ are points of $X$. Then $f(p) \neq f(q)$ for some $f \in \mathcal{F}$, so $f(p)$ and $f(q)$ have disjoint neighborhoods in $Y_f$ such that their inverse images are open and disjoint. Thus $X$ in the weak topology is Hausdorff.
\end{proof}
\end{frame}
\begin{frame}
\frametitle{A metrization theorem}
\begin{theorem}
    If $X$ is a compact topological space, and if some sequence $\{f_n\}$ of continuous real-valued functions separates points on $X$, then $X$ is metrizable.
\end{theorem}
\begin{proof}
    Let $\mathcal{T}$ be the topology of $X$. Without loss of generality, assume $|f_n|\leq 1$ for all $n$. Let $\mathcal{T}_d$ be the topology induced on $X$ by the metric $d(p,q) = \sum_{n=1}^\infty 2^{-n}|f_n(p)-f_n(q)|$. (Check this is indeed a metric.) Note that $d$ is continuous on $X\times X$ by the Weierstrass p-test. The open balls in this metric are $\mathcal{T}$-open, hence $\mathcal{T}_d \subset \mathcal{T}$. Recall the fact on slide 14 to conclude that $\mathcal{T} = \mathcal{T}_d$. Therefore $X$ is metrizable.
\end{proof}
\end{frame}
\begin{frame}
\frametitle{A Lemma}
\begin{lemma}
    Suppose $\Lambda_1,\ldots,\Lambda_n$ and $\Lambda$ are linear functionals on a vector space $X$. Let $N = \{x \in X \mid \Lambda_1 x = \cdots = \Lambda_n x = 0\}$. Then the following are equivalent:\begin{itemize}
        \item[a] There are scalars $\alpha_1,\ldots,\alpha_n$ such that $\Lambda = \alpha_1\Lambda_1,\ldots,\alpha_n\Lambda_n$.
        \item[b] There exists a $\gamma < \infty$ such that $|\Lambda x| \leq \gamma \max_{1\leq i\leq n}|\Lambda_i x|$ for all $x \in X$.
        \item[c] $\Lambda x = 0$ for every $x \in N$.
    \end{itemize}
\end{lemma}
\end{frame}
\begin{frame}
\frametitle{A Lemma}
\begin{proof}
    If you think a while, it is clear that a implies b, b implies c. Thus we prove c implies a. Let $\Phi$ be the scalar field. Define $\pi : X \to \Phi^n$ by $pi(x) = (\Lambda_1 x, \ldots, \Lambda_n x)$. Note that $\pi(x) = \pi(x^\prime)$ implies $\Lambda x = \Lambda x^\prime$. Therefore the linear functional $f$ on $\pi(X)$ is well-defined by $f(\pi(x)) = \Lambda x$. By extension theorems, extend $f$ to a linear functional $F$ on $\Phi^n$. This means there exist $\alpha_i \in \Phi$ such that $F(u_1,\ldots,u_n) = \alpha_1 u_1 + \cdots + \alpha_n u_n$. Therefore $\Lambda x = F(\pi(x)) = F(\Lambda_1 x ,\ldots, \Lambda_n x) = \sum_{i=1}^n \alpha_i \Lambda_i x$.
\end{proof}
\end{frame}
\begin{frame}
\frametitle{A theorem on the topology of dual spaces}
\begin{theorem}
    Suppose $X$ is a vector space, and $X^\prime$ is a vector space of linear functionals on $X$ such that $\Lambda x_1 \neq \Lambda x_2$ for some $\Lambda \in X^\prime$ whenever $x_1\neq x_2$ (i.e. $X^\prime$ is a separating vector space.) Then the topology $\mathcal{T}^\prime$ of $X^\prime$, when applied to $X$, makes $X$ into a locally convex space whose dual space is $X^\prime$.
\end{theorem}
\end{frame}
\end{document}
