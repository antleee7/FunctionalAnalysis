\documentclass{article}
\usepackage[utf8]{inputenc}
\usepackage{amsfonts}
\usepackage{amssymb}
\usepackage{amsthm}
\usepackage{mathtools}
\usepackage[margin=1in]{geometry}
\usepackage{kotex}
\usepackage{scalerel}
\usepackage{tikz-cd}
\usepackage{setspace}
\usepackage{csquotes}
\doublespacing

\newcommand{\ols}[1]{\mskip.5\thinmuskip\overline{\mskip-.5\thinmuskip {#1} \mskip-.5\thinmuskip}\mskip.5\thinmuskip}
\newcommand{\olsi}[1]{\,\overline{\!{#1}}}
\makeatletter
\newcommand\closure[1]{
  \tctestifnum{\count@stringtoks{#1}>1}
  {\ols{#1}}
  {\olsi{#1}}
}
% FROM TOKCYCLE:
\long\def\count@stringtoks#1{\tc@earg\count@toks{\string#1}}
\long\def\count@toks#1{\the\numexpr-1\count@@toks#1.\tc@endcnt}
\long\def\count@@toks#1#2\tc@endcnt{+1\tc@ifempty{#2}{\relax}{\count@@toks#2\tc@endcnt}}
\def\tc@ifempty#1{\tc@testxifx{\expandafter\relax\detokenize{#1}\relax}}
\long\def\tc@earg#1#2{\expandafter#1\expandafter{#2}}
\long\def\tctestifnum#1{\tctestifcon{\ifnum#1\relax}}
\long\def\tctestifcon#1{#1\expandafter\tc@exfirst\else\expandafter\tc@exsecond\fi}
\long\def\tc@testxifx{\tc@earg\tctestifx}
\long\def\tctestifx#1{\tctestifcon{\ifx#1}}
\long\def\tc@exfirst#1#2{#1}
\long\def\tc@exsecond#1#2{#2}
\makeatother


\newtheorem{definition}{Definition}
\newtheorem{example}{Example}
\newtheorem{proposition}{Proposition}
\newtheorem{theorem}{Theorem}



\DeclareMathOperator{\irr}{irr}

\title{하우스도르프의 자살}
\author{수리과학부 이호진}
\date{2023년 6월 15일}

\begin{document}
    \maketitle

    하우스도르프라는 이름은 수학을 공부하는 사람에게는 매우 익숙한 이름이다. 당장 위상수학의 $T2$ 공리도 하우스도르프 공리라 이름이 붙여졌다. 이렇듯 하우스도르프는 수학, 특히 위상수학 및 함수해석의 기초를 닦은 학자였다. 그러나 필자 본인은 인간 하우스도르프에 대해 아는 것이 없음을 깨닫고 하우스도르프에 대해 에세이를 쓰기로 결심했다.

    \section*{하우스도르프의 유년 시절}

    펠릭스 하우스도르프(Felix Hausdorff)는 1868년 프로이센의 브레슬라우(현재는 폴란드의 Wroc\l au)에서 태어났다. 그의 아버지는 루이스 하우스도르프(Louis Hausdorff)로 직물 상인이였으며, 그의 어머니는 헤드위그 티쯔(Hedwig Tietz)였다. 그의 부모는 모두 유대인이었지만, 상당히 부유했기에 하우스도르프는 경제적인 문제를 겪지 않을 수 있었다.

    어린 시절 그는 브레슬라우에서 라이프치히(Leipzig)로 옮겨, 라이프치히에서 유년 시절의 대부분을 보내게 된다. 그는 다방면에 관심이 많은 학생이었으며, 수학 뿐만 아니라 문학과 음악까지 조예가 깊은 학생이었다. 오히려 하우스도르프는 작곡가가 되고 싶어했지만, 부모님의 반대에 부딪혀 결국 수학을 공부하게 된다.

    \section*{수학계의 권위자가 되기까지}

    하우스도르프는 라이프치히 대학교에 입학해서 수학과 천문학을 공부하게 된다. 그의 공부는 주로 대기의 광학적 현상에 대한 것이었다. 그러나 여전히 하우스도르프의 주 관심사는 문학과 철학이었으며, Paul Mongr\'e라는 필명으로 문학 및 철학 활동을 이어 나갔다. Mathematicians under the Nazis라는 책을 쓴 Segal에 의하면, 그에게 수학은 취미 그 이상도 그 이하도 아니었었다.\begin{displayquote}
        \textit{As the child of a wealthy family, he did not have to worry about making a career as a mathematician; for him, mathematics, both as research and as a subject to teach, was more an avocation than anything else.}
    \end{displayquote}

    이렇듯 수학보다 문학, 철학에 관심이 많았던 하우스도르프가 본격적으로 위상수학 및 집합론을 연구하기 시작한 시기는 1904년 이후였다. 그는 이 시기에 처음으로 부분순서집합이라는 개념을 도입하고, 하우스도르프 최대원리를 발표하게 된다. 이는 우리가 잘 알고 있는, 모든 chain이 upper bound를 가지는 partially ordered set은 maximal element을 가진다는 Zorn의 보조정리의 원시적인 형태이다. 또한 이때 칸토어의 연속체 가설을 연구했었다.

    하우스도르프는 1910년까지 라이프치히 대학교에서 가르치다가, 그 이후에 본(Bonn)으로 자리를 옮겼다. 그는 본에서 Eduard Study라는 동료 학자를 만났는데, 이때 스투디는 하우스도르프에게 유명한 문제들을 소개해주며 본격적으로 하우스도르프가 유명해지는 계기를 마련해준다. 본에 잠시 있다가 그는 다시 그라이프스발더로 옮기고, 위상공간과 거리공간의 이론을 담은 \textit{Grundz\"uge der Mengenlehre}를 출판했다. 이 문서는 현재의 위상수학의 토대라고 할 수 있을 정도로 영향력 있는 문서였다. 이렇듯 그는 매우 유명한 수학자가 되었고, 1921년에 다시 본으로 돌아가게 되었다.


    \section*{나치의 집권, 하우스도르프의 최후}


    1920년 2월 24일, 독일에서 국가사회주의 독일 노동자당이 창당되었다. 줄여서 나치, 나치의 집권은 독일 사회를 완전히 바꿔놓았다. 하우스도르프는 아무리 자신이 유대인이라고 하더라도 수학계에서 권위자인 자신에게 해를 끼치지 못할 것이라고 생각했었다. 그러나 안타깝게도 그의 예상은 완전히 빗나갔었다.
 유대인 박해가 날이 갈수록 심해지면서, 대학교의 교수들까지 영향을 미치기 시작했다. 대학교의 교수들은 공공의 이익을 위한다는 명목 하에 히틀러에 대한 충성의 선서를 해야했었다. 선서를 거부한 사람은 교수직에서 박탈되었다. 어쩔 수 없이 하우스도르프도 해당 선서를 읽게 되었다.

    \begin{displayquote}
        \textit{I swear: I will be loyal and obedient to the F\"uhrer of the German Reich and people, Adolf Hitler, observe the laws, and conscientiously fulfill the duties of my office. So help me God.}
    \end{displayquote}

    그러나 박해는 여기서 끝나지 않았다. 그의 수학적인 업적들은 모두 '유대인스럽다', '쓸모없다', '독일답지 않다'는 평가를 듣게 된다. 결국 그는 1935년, 교수직에서 박탈당하게 된다. 독일에서의 수학 활동이 제한되는 바람에 하우스도르프는 폴란드의 저널 \textit{Fundamenta Mathematicae}에서 꾸준하게 활동을 이어나가게 된다. 그러나 이마저도 힘들게 되었다. 1938년, 독일에서 반유대인 폭동인 \textit{Kristallnacht}가 일어나게 되면서 유대인의 체포가 급증하였다. 위기감을 느낀 하우스도르프는 폴랴(George P\'olya)에게 미국의 펠로우십 자리를 부탁해보지만 결국 아무것도 얻어내지 못한다. 그의 몇몇 유대인 동료들은 수용소에 대해 별로 대수롭지 않게 생각했지만, 하우스도르프는 수용소의 실체를 알고 있었다. 결국 그는 수용소행을 피하지 못할 것이라고 생각해서 부인과 함께 동반 자살을 한다.

    \begin{displayquote}
        \textit{Dear friend Wollstein! If you receive these lines, we have solved the problem in a different manner - in the manner of which you have constantly tried to dissuade us. The feeling of security that you have predicted for us once we would overcome the difficulties of the move, is still eluding us, on the contrary, Endenich\footnote{나치 독일의 수용소 캠프} may not even be the end! What has happened in recent months against the Jew evokes justified fear that they will not let us live to see a more bearable situation.}
    \end{displayquote}

    그는 이 유서를 남기고, 아내와 아내의 동생과 함께 수면제를 먹고 자살했다. 여기서 Wollstein은 하우스도르프의 변호사였는데, 그도 얼마 가지 않아 아우슈비츠 수용소에서 생을 마감하게 된다.









    \section*{참고문헌}

\begin{itemize}
    \item[1] Felix Hausdorff - MacTutor Biographies
    \item[2] Felix Hausdorff - JewAge Biographies
    \item[3] Mathematicians under the Nazis, Sanford L. Segal
\end{itemize}


\end{document}